\href{http://www.conan.io/source/jsoncpp/1.8.0/theirix/ci}{\tt }

\href{http://json.org/}{\tt J\+S\+ON} is a lightweight data-\/interchange format. It can represent numbers, strings, ordered sequences of values, and collections of name/value pairs.

Json\+Cpp is a C++ library that allows manipulating J\+S\+ON values, including serialization and deserialization to and from strings. It can also preserve existing comment in unserialization/serialization steps, making it a convenient format to store user input files.

\subsection*{Documentation}

\href{http://open-source-parsers.github.io/jsoncpp-docs/doxygen/index.html}{\tt Json\+Cpp documentation} is generated using \href{http://www.doxygen.org}{\tt Doxygen}.

\subsection*{A note on backward-\/compatibility}


\begin{DoxyItemize}
\item {\ttfamily 1.\+y.\+z} is built with C++11.
\item {\ttfamily 0.\+y.\+z} can be used with older compilers.
\item Major versions maintain binary-\/compatibility.
\end{DoxyItemize}

\subsection*{Contributing to Json\+Cpp}

\subsubsection*{Building and testing with Meson/\+Ninja}

Thanks to David Seifert (), we (the maintainers) now use \href{http://mesonbuild.com/}{\tt meson} and \href{https://ninja-build.org/}{\tt ninja} to build for debugging, as well as for continuous integration (see \href{travis.sh}{\tt {\ttfamily travis.\+sh}} ). Other systems may work, but minor things like version strings might break.

First, install both meson (which requires Python3) and ninja. If you wish to install to a directory other than /usr/local, set an environment variable called D\+E\+S\+T\+D\+IR with the desired path\+: D\+E\+S\+T\+D\+IR=/path/to/install/dir

Then, \begin{DoxyVerb}cd jsoncpp/
BUILD_TYPE=debug
#BUILD_TYPE=release
LIB_TYPE=shared
#LIB_TYPE=static
meson --buildtype ${BUILD_TYPE} --default-library ${LIB_TYPE} . build-${LIB_TYPE}
#ninja -v -C build-${LIB_TYPE} test # This stopped working on my Mac.
ninja -v -C build-${LIB_TYPE}
cd build-${LIB_TYPE}
meson test --no-rebuild --print-errorlogs
sudo ninja install
\end{DoxyVerb}


\subsubsection*{Building and testing with other build systems}

See \href{https://github.com/open-source-parsers/jsoncpp/wiki/Building}{\tt https\+://github.\+com/open-\/source-\/parsers/jsoncpp/wiki/\+Building}

\subsubsection*{Running the tests manually}

You need to run tests manually only if you are troubleshooting an issue.

In the instructions below, replace {\ttfamily path/to/jsontest} with the path of the {\ttfamily jsontest} executable that was compiled on your platform. \begin{DoxyVerb}cd test
# This will run the Reader/Writer tests
python runjsontests.py path/to/jsontest

# This will run the Reader/Writer tests, using JSONChecker test suite
# (http://www.json.org/JSON_checker/).
# Notes: not all tests pass: JsonCpp is too lenient (for example,
# it allows an integer to start with '0'). The goal is to improve
# strict mode parsing to get all tests to pass.
python runjsontests.py --with-json-checker path/to/jsontest

# This will run the unit tests (mostly Value)
python rununittests.py path/to/test_lib_json

# You can run the tests using valgrind:
python rununittests.py --valgrind path/to/test_lib_json
\end{DoxyVerb}


\subsubsection*{Building the documentation}

Run the Python script {\ttfamily doxybuild.\+py} from the top directory\+: \begin{DoxyVerb}python doxybuild.py --doxygen=$(which doxygen) --open --with-dot
\end{DoxyVerb}


See {\ttfamily doxybuild.\+py -\/-\/help} for options.

\subsubsection*{Adding a reader/writer test}

To add a test, you need to create two files in test/data\+:


\begin{DoxyItemize}
\item a {\ttfamily T\+E\+S\+T\+N\+A\+M\+E.\+json} file, that contains the input document in J\+S\+ON format.
\item a {\ttfamily T\+E\+S\+T\+N\+A\+M\+E.\+expected} file, that contains a flatened representation of the input document.
\end{DoxyItemize}

The {\ttfamily T\+E\+S\+T\+N\+A\+M\+E.\+expected} file format is as follows\+:


\begin{DoxyItemize}
\item Each line represents a J\+S\+ON element of the element tree represented by the input document.
\item Each line has two parts\+: the path to access the element separated from the element value by {\ttfamily =}. Array and object values are always empty (i.\+e. represented by either {\ttfamily \mbox{[}\mbox{]}} or {\ttfamily \{\}}).
\item Element path {\ttfamily .} represents the root element, and is used to separate object members. {\ttfamily \mbox{[}N\mbox{]}} is used to specify the value of an array element at index {\ttfamily N}.
\end{DoxyItemize}

See the examples {\ttfamily test\+\_\+complex\+\_\+01.\+json} and {\ttfamily test\+\_\+complex\+\_\+01.\+expected} to better understand element paths.

\subsubsection*{Understanding reader/writer test output}

When a test is run, output files are generated beside the input test files. Below is a short description of the content of each file\+:


\begin{DoxyItemize}
\item {\ttfamily test\+\_\+complex\+\_\+01.\+json}\+: input J\+S\+ON document.
\item {\ttfamily test\+\_\+complex\+\_\+01.\+expected}\+: flattened J\+S\+ON element tree used to check if parsing was corrected.
\item {\ttfamily test\+\_\+complex\+\_\+01.\+actual}\+: flattened J\+S\+ON element tree produced by {\ttfamily jsontest} from reading {\ttfamily test\+\_\+complex\+\_\+01.\+json}.
\item {\ttfamily test\+\_\+complex\+\_\+01.\+rewrite}\+: J\+S\+ON document written by {\ttfamily jsontest} using the {\ttfamily Json\+::\+Value} parsed from {\ttfamily test\+\_\+complex\+\_\+01.\+json} and serialized using {\ttfamily Json\+::\+Styled\+Writter}.
\item {\ttfamily test\+\_\+complex\+\_\+01.\+actual-\/rewrite}\+: flattened J\+S\+ON element tree produced by {\ttfamily jsontest} from reading {\ttfamily test\+\_\+complex\+\_\+01.\+rewrite}.
\item {\ttfamily test\+\_\+complex\+\_\+01.\+process-\/output}\+: {\ttfamily jsontest} output, typically useful for understanding parsing errors.
\end{DoxyItemize}

\subsection*{Using Json\+Cpp in your project}

\subsubsection*{Amalgamated source}

\href{https://github.com/open-source-parsers/jsoncpp/wiki/Amalgamated}{\tt https\+://github.\+com/open-\/source-\/parsers/jsoncpp/wiki/\+Amalgamated}

\subsubsection*{Other ways}

If you have trouble, see the Wiki, or post a question as an Issue.

\subsection*{License}

See the {\ttfamily L\+I\+C\+E\+N\+SE} file for details. In summary, Json\+Cpp is licensed under the M\+IT license, or public domain if desired and recognized in your jurisdiction. 