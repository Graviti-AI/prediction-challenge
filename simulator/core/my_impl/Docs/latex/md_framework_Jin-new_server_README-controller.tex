\subsection*{Header files}

Please include {\ttfamily Controller/\+Controller.\+hpp} before you derive your custome class from {\ttfamily \hyperlink{classController}{Controller}}. 
\begin{DoxyCode}
1 \{C++\}
2 #include "Controller/Controller.hpp"
3 class MyController : public Controller \{
4     // definition body ...
5 \}
\end{DoxyCode}


\subsection*{Construction}

You are supposed to specify the dimension of {\ttfamily state} vector and {\ttfamily input} vector in the argument list of the construction of the base class {\ttfamily \hyperlink{classPlanner}{Planner}}. For example, if we want to derive a {\ttfamily \hyperlink{classHumanCarPlanner}{Human\+Car\+Planner}} class, we should write the construction as follows\+: 
\begin{DoxyCode}
1 \{C++\}
2 MyController::MyController(/* your argument list here */) : Planner(d, 3) \{
3     // construction body ...
4 \}
\end{DoxyCode}
 since we have an intermediate vector of dimension 3 (gas pedal, brake pedal, steering wheel angle), and an input vector of dimension d (the {\ttfamily input} vector is the output of Zeji\textquotesingle{}s planner).

\subsection*{Overriding the virtual method}

All the interface of a {\ttfamily \hyperlink{classController}{Controller}} lies in the virtual method {\ttfamily update}. The signature of this function is\+: 
\begin{DoxyCode}
1 \{C++\}
2 typedef std::vector<double> Vector;
3 Vector HumanCarController::update(Vector input);
\end{DoxyCode}


It takes {\ttfamily input} vector, which is the output of Zeji\textquotesingle{}s planner. The return vector should be the intermediate vector, which is gas pedal, brake pedal, and steering angle.

For each agent, there is a {\ttfamily get\+State()} method, which gives the state vector of this agent. For a car, this state vector has 6 dimensions\+: location x, location y, yaw angle, speed x, speed y, angular velocity of yaw. You can use {\ttfamily agents} to get information of the environment.

\subsection*{Exception handling}

If the dimensions of the vectors do not match, an {\ttfamily std\+::runtime\+\_\+error} will be thrown. For example, if we give a 5-\/dimension vector {\ttfamily state} to our {\ttfamily \hyperlink{classHumanCarPlanner}{Human\+Car\+Planner}}, which expects a 6-\/dimension state vector\+: 
\begin{DoxyCode}
1 \{C++\}
2 try \{
3     Vector currentState = Vector \{0, 0, 0, 1, 2\}; // 5-dimension state vector
4     Vector input = MyPlanner.update(currentState, humanInput, agents); // HumanCar expects 6-dimension
       state vector
5     MyController.update(input);
6 \}
7 catch (std::runtime\_error e) \{
8     std::cout << e.what() << std::endl;
9 \}
\end{DoxyCode}
 